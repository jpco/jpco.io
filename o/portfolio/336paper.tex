\documentclass[letterpaper,12pt]{article}

\title{On the Existence of an Infinite Cube-Free Word on Two Letters}
\author{Jack Conger}

\usepackage{amsmath}
\usepackage{amssymb}
\usepackage{amsthm}
\usepackage{color}

\newcommand{\bl}[1]{\mathbf{#1}}

\newtheorem{thm}{Theorem}
\newtheorem{lem}[thm]{Lemma}

\begin{document}

\maketitle

\section{Introduction}

Axel Thue's first paper on infinite sequences of letters, or words \cite{thue}, establishes some of the first substantial results in the field of combinatorics on words. Often read with another paper published in 1912 \cite{otherthue}, this work establishes possible properties of infinitely long words of alphabets of certain amounts of letters. Particularly, his ultimate theorem in \cite{thue} is the central theorem of this paper: that there exists an infinitely long cube-free word on an alphabet of just two letters.

A main focus of the field of combinatorics on words deals with the nature of long words---what characteristics are unavoidable for words of sufficient length, and which are avoidable. This paper's central theorem is an example of such an investigation: must an infinite word have internal repetition, that is, a square, or a cube, or even greater repetition? This intrigues me personally because it is an elementary example of complexity emerging from a simple system, but such questions have practical ramifications on, for instance, data compression in the field of computer science---can we guarantee certain regularities in arbitrarily large data which we can exploit to compress such data?---and, indeed, combinatorics on words coalesced from results in numerous disparate fields of mathematics, demonstrating it is a field borne of its own utility.

\section{Background Definitions and Results}

The following definitions are summarized from \cite{berstel, lothaire}.

\subsubsection*{Words}

Being a theorem about words, we must define what words \emph{are}.

Consider a finite\footnote{Finititude is not technically necessary, but infinite alphabets are very much out of the scope of this paper.} set $A$. We call $A$ an \emph{alphabet}, with each element of $A$ a \emph{letter}. Any sequence of letters $a_i \in A$ is a \emph{word} over $A$, of the form $a_1 a_2 \cdots a_n$. We define $n$ in this example to be the \emph{length}, or number of characters, in the word. The set of all words over $A$ is denoted by $A^*$. This set includes the \emph{empty word} $\epsilon$, the unique word of length 0. We can also have words of infinite length, such that $u = a_1 a_2 a_3 \cdots$. Infinite words can also be thought of as a mapping from $\mathbb{N} \to A$.

We also define a binary operation \emph{concatenation} such that for finite $u = a_1 a_2 \cdots a_n$ and arbitrary $v = b_1 b_2 \cdots b_m \cdots$ in $A^*$, the concatenation $uv \in A^*$ is
\[
	a_1 a_2 \cdots a_n b_1 b_2 \cdots b_m \cdots.
\]
Note that the empty word $\epsilon$ is the unique word in $A^*$ such that for any (finite) word $u \in A^*$, $u\epsilon = \epsilon u = u$. Note that though the result is symmetric in this case, concatenation is not so in general. However, concatenation \emph{is} associative, a fact which will be of central importance in the proofs to come. Concatenation's property of associativity and the existence of $\epsilon$ classify $A^*$, in algebraic terms, as a monoid. Specifically, the set $A^*$ of an alphabet $A$ is called the \emph{free monoid}.

We define a word $v \in A^*$ to be a \emph{factor} of another word $u \in A^*$ if there exist $x, y \in A^*$ such that $u = xvy$. If $xy = \epsilon$, $u = v$. If $x = \epsilon$, $v$ is a \emph{prefix} of $u$, and if $y = \epsilon$, $v$ is a \emph{suffix} of $u$. A \emph{subword} of a word $u$ is a concatenation of distinct factors of $u$, in the order they appear in $u$.

We define the concatenation of a word with itself $uu = u^2$ to be a \emph{square}, and a word of the form $uuu = u^3$ to be a \emph{cube}. In general, words of the form $uu\cdots u = u^n$ are the $n$th power of $u$. A word is considered to be \emph{square-free} if it contains no factors which are squares---equivalently, a word $u$ is \emph{square-free} if there are no words $x,v,y$ such that $u = xvvy$ and $v \neq \epsilon$. The definition of \emph{cube-free} is analogous.

A \emph{morphism} $f : A^* \rightarrow B^*$ is a function such that
\[
	f(uv) = f(u)f(v), \quad u,v \in A^*
\]
and (as can be easily verified) $f(\epsilon) = \epsilon$. An \emph{isomorphism} is a morphism $f$ with an inverse $f^{-1}$.

\subsubsection*{Codes}

A submonoid $N$ of a monoid $M$ is a subset $N$ of $M$ such that the binary operation of $M$ is closed under $N$, and has the same neutral element $\epsilon$. Given any subset $X$ of our free monoid $A^*$, we can \emph{generate} a submonoid of $X^* \subset A^*$ from the elements of $X$, through repeated concatenation of the elements of $X$. Conversely, for any submonoid $\Sigma$ of $A^*$ there is a unique set $S$ which generates $\Sigma$ and is minimal for set-inclusion---that is, no other set which generates $\Sigma$ is a subset of $S$. In fact, $S$ is computable from $\Sigma$, being
\[
	S = (\Sigma \setminus \epsilon) \setminus (\Sigma \setminus \epsilon)^2;
\]
that is, the set of all nonempty words of $\Sigma$ which cannot be written as the product of two nonempty words of $\Sigma$ (the notation $(\Sigma \setminus \epsilon)^2$ referring to the product of any two nonempty words contained in $\Sigma$). This set $S$ is called the \emph{minimal generating set} of $\Sigma$.

A monoid $M$ is said to be \emph{free} if there is an alphabet $B$ and an isomorphism from the free monoid $B^*$ to $M$. The minimal generating set of a free monoid is called a \emph{code}, or the \emph{basis} of $M$.

A set $X$ is called a \emph{prefix} if no element of $X$ is a prefix of any other. The definition for a \emph{suffix} set is analogous. Any prefix or suffix set is a code.

\section{Proof of Infinite Cube-Free Word}

This section of the paper draws directly from the translation of Thue's paper. 
We begin by proving lemmas similar to our desired result.

\begin{lem} \label{lem:acabcb}
Over a three-letter alphabet $\{a,b,c\}$, there exist arbitrarily 
long square-free words without factors $aca$ or $bcb$.
\end{lem}

\begin{proof}

We begin with an arbitrary square-free word $u$ over $\{a, b, c\}$, and then this construction is carried out in several steps:

First, we replace each occurrence in $u$ of $c$ preceded by $a$ by the word $\beta \alpha$, and each occurrence of $c$ preceded by $b$ by $\alpha \beta$. The resulting word we call $u'$. We can show $u'$ to be square-free by contraposition---first, note that by erasing each $\alpha$ and replacing $\beta$ by $c$, we get $u$ from $u'$. If $u'$ has a square, then by this method, $u$ has a square; thus $u'$ is square-free.

Secondly, we insert $\gamma$ after every letter of $u'$ to get $u''$. Clearly, $u''$ is square-free.

Lastly, we replace any $a$ in $u''$ with $\alpha \beta \alpha$ and any $b$ by $\beta \alpha \beta$ to get $w$. This is our final word.

It remains to show $w$ is square-free and does not contain $\alpha \gamma \alpha$ or $\beta \gamma \beta$. For the first fact, note that in $u'$, all $a$ or $\alpha$ alternate with all $b$ or $\beta$. Thus, in $u''$, the three-letter factors with $\gamma$ as the second letter are $a\gamma b$, $a\gamma\beta$, $\alpha\gamma b$, or $\alpha\gamma\beta$. Thus, in $w$ the only such factors are $\alpha\gamma\beta$ and $\beta\gamma\alpha$.

Now we show $w$ is square-free by contradiction. Suppose there is a square $ss$ in $w$. Since the only factors between any two $\gamma$ are $\alpha$, $\beta$, $\alpha\beta\alpha$, and $\beta\alpha\beta$, there must in be at least one $\gamma$ in $ss$ and thus $s$. If there is only one $\gamma$ and is not the last letter in $s$, assume without loss of generality that $\alpha$ follows $\gamma$ (the argument for $\beta$ is the same.) Then $ss$ contains $\gamma\alpha\gamma\alpha$ or $\gamma\alpha\beta\alpha\gamma\alpha$, where in both cases $w$ has a factor $\alpha\gamma\alpha$, contradiction.

Thus $s$ must contain at least two $\gamma$. Then, for $X = \{\alpha, \beta, \alpha\beta\alpha, \beta\alpha\beta\}$, we express $s$ in the form
\[
	s = p\gamma x_1 \gamma \cdots \gamma x_m \gamma q, \quad x_j \in X, qp \in X, m \geq 1.
\]
If $q = \epsilon$, then $p \in X$, and if we undo step three of our construction, we get a square in $u''$; the same holds for $p = \epsilon$. Thus $q, p \neq \epsilon$, and $qp = \alpha\beta\alpha$ or $\beta\alpha\beta$. Let us assume the first case (again, we have the same argument for the alternative). Then $(q,p) = (\alpha, \beta\alpha)$, or $(q,p) = (\alpha\beta, \alpha)$. These are symmetric---the following argument is generally the same for the end of $u$ in the second case as it is for the beginning of $u$ in the first case. Take the first case again without loss of generality. By construction, $w$ cannot start with $\beta\alpha\gamma$, so we must have at least an $\alpha$ preceding $ss$ in $w$, which means that $w$ contains $qp\gamma x_1 \gamma \cdots \gamma x_m \gamma$ as a factor, but this as before implies $u''$ has a square. Thus we have a contradiction, so in the end $w$ must be square-free.

\end{proof}

\begin{lem} There exists a sequence $(w_n)_{n \geq 0}$ of square-free words 
over three letters such that $w_n$ is a prefix of $w_{n+1}$. Stated otherwise, there is an infinitely long square-free word over three letters.\end{lem}

\begin{proof}

We use another construction as for Lemma~\ref{lem:acabcb}, which is much more concise but much less illuminating. For any $u \in \{a, b, c\}^*$ with no factors $aca$ or $bcb$, we create a new word $\sigma(u)$ by the function
\[
	\sigma :
	\begin{array}{ll}
		a \rightarrow abac & \\
		b \rightarrow babc & \\
		c \rightarrow bcac & \text{ if $c$ is preceded by $a$} \\
		c \rightarrow acbc & \text{ if $c$ is preceded by $b$}.
	\end{array}
\]
This produces the same result as the previous construction: however, with this, we can more easily apply $\sigma$ an arbitrary number of times to $u$ to get a square-free word on $\{a, b, c\}^*$ of arbitrarily great length, and therefore an infinitely long word.

\end{proof}

This, along with certain algebraic features of words, leads directly to our result.

\begin{thm}
There exists an infinite cube-free word over two letters.
\end{thm}

\begin{proof}

Our proof begins with a square-free infinite word $u \in \{x, y, z\}^*$. We create a morphism
\[
	f :
	\begin{array}{l}
		x \rightarrow a \\
		y \rightarrow ab \\
		z \rightarrow abb,
	\end{array}
\]
and assert that $f(u) \in \{a, b\}^*$ is cube-free.

Consider $X = \{a, ab, abb\}$. This is a suffix code, giving us the following chain of results.

\begin{enumerate}
\item If $u$ and $v$ are words on $\{x, y\}$ such that $f(u) = f(v)$, then $u = v$.

\item The morphism $f$ is injective.

\begin{proof}
This holds because $X$ is a code.
\end{proof}

Now let $\bl{x}$ be an infinite square-free word over the letters $x$, $y$, and $z$, and let $\bl{y} = f(\bl{x})$.

\item If $\bl{y}$ contains the factor $uuu$ (that is, a cube,) then $u$ does not start with the letter $a$.

\begin{proof}
If $u$ starts with the letter $a$, then there is a (unique) factor $v$ of $\bl{x}$ such that $f(v)=u$. But then $\bl{x}$ contains the square $vv$, contradiction.
\end{proof}

\item If $\bl{y}$ contains a factor $uuu$, then $u$ does not begin with $bb$. 

\begin{proof}
	If $u$ begins with $bb$, then any occurence of $u$ is preceded by an $a$, and therefore also $u$ ends with an $a$. Thus, setting $u = u'a$, the word $\bl{y}$ has a factor $au'au'au'$, contrary to the preceding lemma.
\end{proof}

\item If $y$ contains the factor $uuu$, then $u$ does not end with $b$.

\begin{proof}
	By the preceding lemmas, $u$ must begin with $ba$, and suppose $u$ ends with $b$, such that $u = bau'b$. Then $\bl{y}$ contains a factor $bau'bbau'bbau'b$, which means $\bl{y}$ contains the factor $au'bbau'bb$, meaning (by the fact $f$ is injective) $\bl{x}$ contains a square, contradiction.
\end{proof}
\end{enumerate}

This is sufficient to prove the theorem. If $\bl{y}$ includes $uuu$, then $u$ must start with $ba$ and end with $a$. If $u = ba$, then $\bl{y}$ contains $bababa$, which means $\bl{x}$ contains the square $yy$. Otherwise, if $u = bau'a$ for some $u'$, then $\bl{y}$ contains $bau'abau'abau'a$ and thus contains $abau'abau'$, meaning $\bl{x}$ has a square, contradiction.

Thus $\bl{y}$ contains no factors of the form $uuu$, and is therefore square-free, which is what we wanted.

\end{proof}

\section{Further Applications}

In a direct sense, the proof of infinite square-free (and/or cube-free in cases) words has applications in the Burnside problem in abstract algebra (Is every finitely generated torsion semigroup finite?)\cite{lothaire}

Further, the square-free word constructed in Lemma 2 as well as the cube-free word constructed in the final theorem were both independently demonstrated and proven by Morse in later decades; the latter word is known now as the Thue-Morse sequence and has had applications in several fields of number theory and combinatorics even before Thue first codified the sequence. Investigations of square-free words and generalizations thereof (often $k$th-power-free words, or theorems on the distance between repeated factors of infinite words) have been further developed throughout the 20th century.

\begin{thebibliography}{9}

\bibitem{berstel}
	J. Berstel,
	\emph{Axel Thue's papers on repetitions in words: a translation}.
	Laboratoire de combinatoire et d'informatique math\'ematique, Montreal, Qc.,
	1995.

\bibitem{lothaire}
	M. Lothaire,
	\emph{Combinatorics on Words}.
	Encyclopedia of Mathematics and Its Applications.
	Addison-Wesley, Reading, Massachusetts,
	1983

\bibitem{thue}
	A. Thue,
	\"Uber unendliche Zeichenreihen,
	\emph{Kra. Vidensk. Selsk. Skrifter. I.
	Mat.-Nat. Kl.,}
	Christianna 1906, Nr. 7.

\bibitem{otherthue}
	A. Thue,
	\"Uber die gegenseitige Lage gleicher Teile gewisser Zeighenreihen,
	\emph{Kra. Vidensk. Selsk. Skrifter. I.
	Mat.-Nat. Kl.,}
	Christiana 1912, Nr. 10.

\end{thebibliography}

\end{document}
